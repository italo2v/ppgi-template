% ---
% RESUMOS
% ---

% RESUMO em português
\setlength{\absparsep}{18pt} % ajusta o espaçamento dos parágrafos do resumo
\begin{resumo}

	Araujo, Marcos V. M. \textbf{\imprimirtitulo}. \pageref{LastPage} f. \imprimirtipotrabalho - Instituto de Matemática, Instituto Tércio Pacitti, Universidade Federal do Rio de Janeiro. \imprimirlocal. \imprimirdata. 
    
Os \textit{smart grids} produzem um grande volume de dados originados dos \textit{smart meters}, atuadores da rede elétrica e da geração de energia. Diante disso, a aquisição dos dados, o processamento em tempo real para demanda de carga e faturamento apresentam desafios como armazenar e processar um grande volume de dados. Contudo, há também desafios de segurança da informação pois os dados gerados por \textit{smart meters} são trafegados na rede os quais podem ocorrer ataques de interceptação, modificação ou fabricação de dados. Para dar respostas aos desafios de \textit{smart grids}, este trabalho propõe um processamento seguro dos dados de consumo dos \textit{smart meters} por um ambiente de computação em nuvem seguro. A proposta protege os dados manipulados pelas concessionárias para gerar informações do faturamento e do balanço de carga. No desenvolvimento do processo de segurança dos dados, a computação em nuvem é utilizada pois tem a característica de lidar com um grande volume de dados com um menor custo monetário e de forma segura. Além do mais, para garantir confidencialidade e integridade dos dados no ambiente em nuvem foram utilizados processadores com a tecnologia SGX, os quais permitem instanciar áreas de memórias protegidas. Não obstante, em vista de otimizar a arquitetura proposta a mesma foi desenvolvida por meio de programação baseada em microserviços que é uma abordagem para construção de aplicações nos quais são desenvolvidos pequenos serviços onde cada serviço executa em um processo de forma independente gerando uma aplicação fracamente acoplada e mais simples para desenvolver, gerenciar e escalar. Para a proposta, alguns serviços rodarão dentro de enclaves SGX na nuvem para garantir serviços seguros nesse ambiente. Adicionalmente a segurança, é agregado uma abordagem com terceira parte confiável para auditar as funcionalidades de faturamento, balanço de carga. Por fim, para validar a proposta são discutidos 3 cenários de experimentos: (\textit{i}) a proposta foi desenvolvida em arquitetura monolítica; (\textit{ii}) a proposta foi desenvolvida em arquitetura baseada em microserviços; e (\textit{iii}) a proposta foi desenvolvida em arquitetura baseada em microserviços com o uso do SGX. Embora nos experimentos o uso do SGX apresentou um aumento no tempo de execução do processamento, há ganho ao usar o SGX em relação a segurança de aplicações que executam em um ambiente não confiável.

 \textbf{Palavras-chaves}: Computação em Nuvem, Redes Inteligentes, Segurança da Informação, Intel SGX, Microserviços.
\end{resumo}

% _________________________________________________________________
% ABSTRACT in english
\begin{resumo}[Abstract]
 \begin{otherlanguage*}{english}
   
   
   Araujo, Marcos V. M. \textbf{\imprimirtitulo}. \pageref{LastPage} f. \imprimirtipotrabalho - Instituto de Matemática, Instituto Tércio Pacitti, Universidade Federal do Rio de Janeiro. \imprimirlocal. \imprimirdata.

   Smart grids produce a large volume of data from smart meters, power grid actuators and power generation. Given this, data acquisition, real-time processing for load demand and billing presents challenges such as storing and processing a large volume of data. However, there are also challenges to information security, since data generated by smart meters is traversed in the network, which can cause interception, modification or fabrication attacks. In order to respond to the challenges of smart grids, this work proposes secure processing of smart meter consumption data by cloud computing environment. The proposal protects data handled by utilities to generate billing and freight balance information. In developing the data security process, cloud computing is used because it has the characteristic of handling a large volume of data, with a lower monetary cost and in a safe way. In addition, to ensure confidentiality and data integrity in the cloud environment, processors using SGX technology were used to instantiate protected memory areas. However, in order to optimize the proposed architecture, it has been developed through micro-service-based programming, which is an approach for building applications, in which small services are developed where each service executes in a process independently generating a weak application coupled and simpler to develop, manage and scale. For the proposal, some services will run within SGX enclaves in the cloud to ensure secure services in that environment. In addition to security, a trusted third party approach is added to audit billing and freight balance functionality. Finally, to validate the proposal are discussed 3 scenarios of experiments: (\textit{i}) the proposal was developed in monolithic architecture; (\textit{ii}) the proposal was developed in architecture based on microservices; and (\textit{iii}) the proposal was developed in a microservice-based architecture with the use of SGX. Although in the experiments the use of the SGX presented an increase in the execution time of the processing, there is gain when using the SGX in relation to the security of applications that execute in an unreliable environment.

   \vspace{\onelineskip}
 
   \noindent 
   
   \textbf{Keywords}: Cloud Computing, Smart Grid, Information Security, Intel SGX 
 \end{otherlanguage*}
\end{resumo}
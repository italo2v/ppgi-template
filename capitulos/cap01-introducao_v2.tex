\OnehalfSpacing
% ---
\chapter{Introdução}
% ---

Os \textit{Smart Grids} (SG) são sistemas que combinam a distribuição de energia elétrica tradicional com tecnologias de computação e telecomunicação para compor um sistema mais robusto e eficiente no supervisionamento e na distribuição de energia elétrica. Com esse sistema é possível monitorar, analisar e controlar a rede de distribuição de energia elétrica e a comunicação com os consumidores \cite{Barbosa2016}. Uma característica importante dos SGs é ser constituído por um conjunto de tecnologias e dispositivos, no qual esse conjunto é denominado de \textit{Advanced Metering Infrastructure} (AMI).  
um estilo de arquitetura que permite estruturar uma aplicação como uma coleção de serviços fracamente acoplados. Em uma arquitetura de microserviços, os serviços devem ser atômicos, ou seja, fazem somente uma simples operação e os protocolos de comunicação devem ser leves \cite{Hasselbring2016}. O benefício de decompor uma aplicação em diferentes serviços menores permite a modularidade e torna a aplicação mais fácil de entender, desenvolver e testar \cite{Richardson-Chris}. Além disso, na proposta a granularidade dos serviços permitiu que serviços seguros sejam responsáveis em deter apenas as chaves necessárias apenas para o seu processo. 

% --------------------------------------------------------
%\section{Objetivos}
% --------------------------------------------------------
%Essa dissertação tem como objetivo principal propor uma aplicação para aquisição e autenticação dos dados de consumo gerados por SMs usando uma nuvem computacional.

%Com objetivos secundários:

%\begin{itemize}
%	\item Verificar a pertinência do uso da arquitetura SGX de forma a satisfazer os requisitos de segurança da informação;

%	\item Propor um abordagem para garantir a medição até o faturamento;
    
 %   \item Anonimizar os dados de consumo por uma terceira parte confiável;

%\end{itemize}
% ---

%\section{Relevância}
% ---
%Os desafios de segurança para aplicações em nuvem de coleta de dados, ainda são problemas abertos, carente de soluções definitivas.

%O trabalho colaborá diminuindo os obstáculos em aplicações de computação em nuvem empregadas em adquirir e processar dados de medidores inteligentes. Contribuindo com adoção célere de REI seguras.
% ---

\section{Organização do Trabalho}
% ---
Esta seção do trabalho apresenta a estrutura no qual o trabalho está organizado.

O capitulo 2 apresenta as bases teorias da literatura mundial que concerne esta pesquisa. Tem bases teóricas referentes a computação em nuvem, \textit{Smart Grids}, \textit{Smart Meters}, microsserviços e a tecnologia Intel SGX. 

O capitulo 3 apresenta trabalhos relacionados a esta pesquisa. Os trabalhos serão referentes a segurança em ambientes em computação em nuvem e em mecanismos de segurança utilizando a plataforma Intel SGX.

%O capítulo 7 apresenta o cronograma para término dessa dissertação de mestrado.

% ---

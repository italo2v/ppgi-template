% ---------------------------------------------------------------------------
% ---------------------------------------------------------------------------
% Modelo LaTex para preparação do documento final de Dissertação de Mestrado
% O modelo está em conformidade com ABNT NBR 14724:2011: 
% Programa de Pós-Graduação em Informática
% Universidade Federal do Rio de Janeiro
%LABNET - Laboratorio de redes e multimidias
%Qualquer alteracao deve-se entrar em contato para mudancas
%marcosaraujo.mvma@gmail.com
% Versão: v0.9
% ---------------------------------------------------------------------------
% ---------------------------------------------------------------------------

\documentclass[
	% -- opções da classe memoir --
	12pt,					% tamanho da fonte
	%openright,				% capítulos começam em pág ímpar (insere página vazia caso preciso)
	oneside,					% para impressão em verso e anverso. Oposto a oneside
	a4paper,					% tamanho do papel. 
	% -- opções da classe abntex2 --
	chapter=TITLE,			% títulos de capítulos convertidos em letras maiúsculas
	section=TITLE,			% títulos de seções convertidos em letras maiúsculas
	%subsection=TITLE,		% títulos de subseções convertidos em letras maiúsculas
	%subsubsection=TITLE,	% títulos de subsubseções convertidos em letras maiúsculas
	% -- opções do pacote babel --
	english,					% idioma adicional para hifenização
	%french,					% idioma adicional para hifenização
	%spanish,				% idioma adicional para hifenização
	brazil					% o último idioma é o principal do documento
	]{abntex2}
% ---------------------
% Pacotes OBRIGATÓRIOS
% ---------------------
\OnehalfSpacing
%\usepackage{lmodern}				% Usa a fonte Latin Modern
%\usepackage{fontspec}
%\setmainfont{Arial}		
\usepackage[T1]{fontenc}		% Selecao de codigos de fonte.
\usepackage{uarial}
\usepackage[utf8]{inputenc}		% Codificacao do documento (conversão automática dos acentos)
\usepackage{lastpage}			% Usado pela Ficha catalográfica
\usepackage{indentfirst}			% Indenta o primeiro parágrafo de cada seção.
\usepackage{color}				% Controle das cores
\usepackage{graphicx,graphicx}	% Inclusão de gráficos
\usepackage{epsfig,subfig}		% Inclusão de figuras
\usepackage{microtype} 			% Melhorias de justificação
%\usepackage[top=3cm, bottom=2cm, left=3cm, right=2cm]{geometry}
% ---------------------
		
% ---------------------
% Pacotes ADICIONAIS
% ---------------------
\usepackage{lipsum}						% Geração de dummy text
\usepackage{amsmath,amssymb,mathrsfs}	% Comandos matemáticos avançados 
\usepackage{setspace}  					% Para permitir espaçamento simples, 1 1/2 e duplo
\usepackage{verbatim}					% Para poder usar o ambiente "comment"
\usepackage{tabularx} 					% Para poder ter tabelas com colunas de largura auto-ajustável
\usepackage{afterpage} 					% Para executar um comando depois do fim da página corrente
\usepackage{url} 						% Para formatar URLs (endereços da Web)
\usepackage{lscape}						% para pode usar uma página 90 graus
\usepackage{multirow}					% para pode usar multiplas linhas em tabelas
\usepackage[section]{placeins}			% para não pular de secção as figuras e nem ass tabelas
% ---------------------

% ---------------------
% Pacotes de CITAÇÕES
% ---------------------
\usepackage[brazilian,hyperpageref]{backref}	% Paginas com as citações na bibl
\usepackage[alf]{abntex2cite}				% Citações padrão ABNT (alfa)
%\usepackage[num]{abntex2cite}				% Citações padrão ABNT (numericas)
% ---------------------

% Configurações de CITAÇÕES para abntex2
\include{extras/conf_citacoes}

% Inclusão de dados para CAPA e FOLHA DE ROSTO (título, autor, orientador, etc.)
% ---
% Informações de dados para CAPA e FOLHA DE ROSTO
% ---
\titulo{Seu titulo}
\autor{Seu nome}
\local{Rio de Janeiro}
\data{2017}
\orientador{Seu orientador, Dr.}
\coorientador{Seu coorientador, DSc.}
\instituicao{%
  Universidade Federal do Rio de Janeiro
  \par
  Instituto de Matemática
  \par
  Instituto Tércio Pacitti de Aplicações e Pesquisas Computacionais
  \par
  Programa de Pós-Graduação em Informática}
\tipotrabalho{Dissertação (Mestrado)}
% O preambulo deve conter o tipo do trabalho, o objetivo,
% o nome da instituição e a área de concentração
\preambulo{\textbf{Qualificação de Mestrado} apresentada ao Programa de Pós-Graduação em Informática, Instituto de Matemática e Instituto Tércio Parcitti da Universidade Federal do Rio de Janeiro(área de concentração: Redes de Computadores e Sistemas Distribuídos), como parte dos requisitos necessários para a obtenção do Título de Mestre em Informática.}
% ---


% Inclui Configurações de aparência do PDF Final
%  Configurações de aparência do PDF final
% NÃO ALTERAR!!!

% alterando o aspecto da cor azul
\definecolor{blue}{RGB}{41,5,195}

% informações do PDF
\makeatletter


%%%%%%%%%%%%%%%%%%%%%%%%%%%%%%%%%%%%%%%%%%%%%%%%%%%%%%%%%%%%%%%%%%%
%% COLOCANDO SECAO PRIMARIA E SECUNDARIA COMO CAIXA ALTA %%
%%%%%%%%%%%%%%%%%%%%%%%%%%%%%%%%%%%%%%%%%%%%%%%%%%%%%%%%%%%%%%%%%%%

\let\oldcontentsline\contentsline

\def\contentsline#1#2{%
	\expandafter\ifx\csname l@#1\endcsname\l@chapter
	\expandafter\@firstoftwo
	\else
	\expandafter\@secondoftwo
	\fi
	{%
		\oldcontentsline{#1}{\MakeTextUppercase{#2}}%
	}{%
	\expandafter\ifx\csname l@#1\endcsname\l@section
	\expandafter\@firstoftwo
	\else
	\expandafter\@secondoftwo
	\fi
	{%
		\oldcontentsline{#1}{\MakeTextUppercase{#2}}%
	}{%
	\oldcontentsline{#1}{#2}%
  }%
}%
}

%\settocpreprocessor{section}{%
%    \let\tempf@rtoc\f@rtoc%
%    \def\f@rtoc{%
%     \texorpdfstring{\MakeTextUppercase{%
%        \tempf@rtoc}%
%      }{\tempf@rtoc}%
%    }%
%}
%%%%%%%%%%%%%%%%%%%%%%%%%%%%%%%%%%%%%%%%%%%%%%%%%%%%%%%%%%%%%%%%%%%


\hypersetup{
     	%pagebackref=true,
		pdftitle={\@title}, 
		pdfauthor={\@author},
    		pdfsubject={\imprimirpreambulo},
	    pdfcreator={LaTeX with abnTeX2},
		pdfkeywords={abnt}{latex}{abntex}{abntex2}{trabalho acadêmico}, 
		colorlinks=true,       		% false: boxed links; true: colored links
    		linkcolor=black,          	% color of internal links
    		citecolor=black,        		% color of links to bibliography
    		filecolor=black,      		% color of file links
		urlcolor=blue,
		bookmarksdepth=4
} 
\makeatother
% --- 


% O tamanho da identação do parágrafo é dado por:
\setlength{\parindent}{1.3cm}

% Controle do espaçamento entre um parágrafo e outro:
\setlength{\parskip}{0.2cm}  % tente também \onelineskip


\setlrmarginsandblock{3cm}{2cm}{*}
\setulmarginsandblock{3cm}{2cm}{*}
\checkandfixthelayout

% ---------------------
% Compila o indice
% ---------------------
\makeindex
% ---------------------

%%%%%%%%%%%%%%%%%%%%%%%%%%%
%%  INICIO DO DOCUMENTO  %%
%%%%%%%%%%%%%%%%%%%%%%%%%%%
\begin{document}

% Retira espaço extra obsoleto entre as frases.
\frenchspacing

% ----------------------------------------------------------
% ELEMENTOS PRÉ-TEXTUAIS (Capa, Resumo, Abstract, etc.) 
% ---------------------------------------------------------- 
\pretextual

% Capa
% ---
% Impressão da Capa
% ---
  \begin{capa}%
    \begin{figure}[h!]%
        \centering%
        \includegraphics[scale=1]{figs/banner.png}
      \end{figure}%
    \center
	\ABNTEXchapterfont\large{Universidade Federal do Rio de Janeiro\\
    Instituto de Matemática\\
    Instituto Tércio Pacitti de Aplicações e Pesquisas Computacionais\\
    Programa de Pós-Graduação em Informática}
	%\vspace{1.2cm}

    \vfill
    \ABNTEXchapterfont\bfseries\LARGE\imprimirtitulo
    \vfill

	%\vfill
	\ABNTEXchapterfont\large\imprimirautor
	\vfill
%
	%úmero de Ordem PPGCC: M001
	
    \large\imprimirlocal
    \\\large\imprimirdata

    \vspace*{1cm}
  \end{capa}
% ---

% Folha de rosto (o * indica que haverá a ficha bibliográfica)
\imprimirfolhaderosto*

% Imprimir Ficha Catalografica
\include{pretextual/catalografica}

% Inserir Folha de Aprovação
% ---
% Assinaturas
% ---
% Isto é um exemplo de Folha de aprovação, elemento obrigatório da NBR
% 14724/2011 (seção 4.2.1.3). Você pode utilizar este modelo até a aprovação
% do trabalho. Após isso, substitua todo o conteúdo deste arquivo por uma
% imagem da página assinada pela banca com o comando abaixo:
%
% \includepdf{folhadeaprovacao_final.pdf}
%
\begin{folhadeaprovacao}

  \begin{center}
    {\ABNTEXchapterfont\large\imprimirautor}

    \vspace*{\fill}\vspace*{\fill}
    \begin{center}
      \ABNTEXchapterfont\bfseries\Large\imprimirtitulo
    \end{center}
    \vspace*{\fill}
    
    \hspace{.45\textwidth}
    \begin{minipage}{.5\textwidth}
        \imprimirpreambulo
    \end{minipage}%
    \vspace*{\fill}
   \end{center}
        
   Trabalho aprovado. \imprimirlocal, 02 de Outubro de 2017:

   \assinatura{\textbf{\imprimirorientador} \\ Orientador} 
   \assinatura{\textbf{\imprimircoorientador} \\ Coorientador} 
   \assinatura{\textbf{Membro 1, DSc} \\}
   \assinatura{\textbf{Membro 2} \\ Convidado 2}
   \assinatura{\textbf{Membro 3} \\ Convidado 3}
%   \begin{center}
%    \vspace*{0.1cm}
%    {\large\imprimirlocal}
%    \par
%    {\large\imprimirdata}
%    \vspace*{1cm}
%  \end{center}
  
\end{folhadeaprovacao}
% ---


% Dedicatória
%\include{pretextual/dedicatoria}

% Agradecimentos
%\include{pretextual/agradecimentos}

% Epígrafe
% ---
% Epígrafe
% ---
\begin{epigrafe}
    \vspace*{\fill}
	\begin{flushright}
		\textit{``Frase.''\\
		          (Autor)}
	\end{flushright}
\end{epigrafe}
% ---


% Resumo e Abstract
% ---
% RESUMOS
% ---

% RESUMO em português
\setlength{\absparsep}{18pt} % ajusta o espaçamento dos parágrafos do resumo
\begin{resumo}

	Araujo, Marcos V. M. \textbf{\imprimirtitulo}. \pageref{LastPage} f. \imprimirtipotrabalho - Instituto de Matemática, Instituto Tércio Pacitti, Universidade Federal do Rio de Janeiro. \imprimirlocal. \imprimirdata. 
    
Os \textit{smart grids} produzem um grande volume de dados originados dos \textit{smart meters}, atuadores da rede elétrica e da geração de energia. Diante disso, a aquisição dos dados, o processamento em tempo real para demanda de carga e faturamento apresentam desafios como armazenar e processar um grande volume de dados. Contudo, há também desafios de segurança da informação pois os dados gerados por \textit{smart meters} são trafegados na rede os quais podem ocorrer ataques de interceptação, modificação ou fabricação de dados. Para dar respostas aos desafios de \textit{smart grids}, este trabalho propõe um processamento seguro dos dados de consumo dos \textit{smart meters} por um ambiente de computação em nuvem seguro. A proposta protege os dados manipulados pelas concessionárias para gerar informações do faturamento e do balanço de carga. No desenvolvimento do processo de segurança dos dados, a computação em nuvem é utilizada pois tem a característica de lidar com um grande volume de dados com um menor custo monetário e de forma segura. Além do mais, para garantir confidencialidade e integridade dos dados no ambiente em nuvem foram utilizados processadores com a tecnologia SGX, os quais permitem instanciar áreas de memórias protegidas. Não obstante, em vista de otimizar a arquitetura proposta a mesma foi desenvolvida por meio de programação baseada em microserviços que é uma abordagem para construção de aplicações nos quais são desenvolvidos pequenos serviços onde cada serviço executa em um processo de forma independente gerando uma aplicação fracamente acoplada e mais simples para desenvolver, gerenciar e escalar. Para a proposta, alguns serviços rodarão dentro de enclaves SGX na nuvem para garantir serviços seguros nesse ambiente. Adicionalmente a segurança, é agregado uma abordagem com terceira parte confiável para auditar as funcionalidades de faturamento, balanço de carga. Por fim, para validar a proposta são discutidos 3 cenários de experimentos: (\textit{i}) a proposta foi desenvolvida em arquitetura monolítica; (\textit{ii}) a proposta foi desenvolvida em arquitetura baseada em microserviços; e (\textit{iii}) a proposta foi desenvolvida em arquitetura baseada em microserviços com o uso do SGX. Embora nos experimentos o uso do SGX apresentou um aumento no tempo de execução do processamento, há ganho ao usar o SGX em relação a segurança de aplicações que executam em um ambiente não confiável.

 \textbf{Palavras-chaves}: Computação em Nuvem, Redes Inteligentes, Segurança da Informação, Intel SGX, Microserviços.
\end{resumo}

% _________________________________________________________________
% ABSTRACT in english
\begin{resumo}[Abstract]
 \begin{otherlanguage*}{english}
   
   
   Araujo, Marcos V. M. \textbf{\imprimirtitulo}. \pageref{LastPage} f. \imprimirtipotrabalho - Instituto de Matemática, Instituto Tércio Pacitti, Universidade Federal do Rio de Janeiro. \imprimirlocal. \imprimirdata.

   Smart grids produce a large volume of data from smart meters, power grid actuators and power generation. Given this, data acquisition, real-time processing for load demand and billing presents challenges such as storing and processing a large volume of data. However, there are also challenges to information security, since data generated by smart meters is traversed in the network, which can cause interception, modification or fabrication attacks. In order to respond to the challenges of smart grids, this work proposes secure processing of smart meter consumption data by cloud computing environment. The proposal protects data handled by utilities to generate billing and freight balance information. In developing the data security process, cloud computing is used because it has the characteristic of handling a large volume of data, with a lower monetary cost and in a safe way. In addition, to ensure confidentiality and data integrity in the cloud environment, processors using SGX technology were used to instantiate protected memory areas. However, in order to optimize the proposed architecture, it has been developed through micro-service-based programming, which is an approach for building applications, in which small services are developed where each service executes in a process independently generating a weak application coupled and simpler to develop, manage and scale. For the proposal, some services will run within SGX enclaves in the cloud to ensure secure services in that environment. In addition to security, a trusted third party approach is added to audit billing and freight balance functionality. Finally, to validate the proposal are discussed 3 scenarios of experiments: (\textit{i}) the proposal was developed in monolithic architecture; (\textit{ii}) the proposal was developed in architecture based on microservices; and (\textit{iii}) the proposal was developed in a microservice-based architecture with the use of SGX. Although in the experiments the use of the SGX presented an increase in the execution time of the processing, there is gain when using the SGX in relation to the security of applications that execute in an unreliable environment.

   \vspace{\onelineskip}
 
   \noindent 
   
   \textbf{Keywords}: Cloud Computing, Smart Grid, Information Security, Intel SGX 
 \end{otherlanguage*}
\end{resumo}

% Lista de ilustrações
\pdfbookmark[0]{\listfigurename}{lof}
\listoffigures*
\cleardoublepage

% Lista de tabelas
\pdfbookmark[0]{\listtablename}{lot}
\listoftables*
\cleardoublepage

% Lista de abreviaturas e siglas
\begin{siglas}

  \item[AMI]\textit{Advanced Metering Infrastructure}
  \item[CIF]\textit{Cloud Industry Forum}
  \item[CGEE]{Centro De Gestão e Estudos Estratégicos}
  \item[CPU]{Central Processing Unit}
  \item[CSA]\textit{Cloud Security Alliance} 
  \item[ECP]\textit{European Cloud Partnership}
  \item[IaaS]\textit{Infrastructure as a Service}
  \item[INMETRO]{Instituto Nacional de Metrologia, Qualidade e Tecnologia}
  \item[PaaS]\textit{Platform as a Service}
  \item[RDIC]Remote data integrity checking
  \item[SaaS]\textit{Software as a Service}
  \item[SCONE]\textit{Security Linux Containers with Intel SGX}
  \item[SG]\textit{Smart Grid}
  \item[SGX]\textit{Software Guard Extensions}
  \item[SM]\textit{Smart Meter}
  \item[SSL]\textit{Secure Sockets Layer}
  \item[TaLos]\textit{Secure and Transparent TLS Termination inside SGX}
  \item[TLS]\textit{Transport Layer Security}
  \item[UE]{União Europeia}
  
\end{siglas}

% Lista de símbolos
%\begin{simbolos}
%  \item[$ \Gamma $] Letra grega Gama
%  \item[$ \Lambda $] Lambda
%  \item[$ \zeta $] Letra grega minúscula zeta
%  \item[$ \in $] Pertence
%\end{simbolos}

% Inserir o SUMÁRIO
\pdfbookmark[0]{\contentsname}{toc}
\tableofcontents*
\cleardoublepage
%\settocdepth{chapter}

% ----------------------------------------------------------
% ELEMENTOS TEXTUAIS (Capítulos)
% ----------------------------------------------------------
\textual
% Elementos textuais com numeração arábica
\pagenumbering{arabic}
% Reinicia a contagem do número de páginas
\setcounter{page}{13}

% Inclui cada capitulo da Dissertação
%\include{capitulos/introducao}

% PARTE - Define a divisão do documento em partes (Não é obrigatório)
%\part{Preparação da pesquisa}
\OnehalfSpacing
% ---
\chapter{Introdução}
% ---

O artigo \cite{Wang2013} ...

A Figura \ref{fig:smart_meter} ilustra ...


\begin{figure}[ht]
\begin{center}
    \includegraphics[width=0.5\textwidth]{figs/smartmeter.png}
\end{center}
\caption{Titulo da Figura}
\label{fig:smart_meter}
\end{figure}

% --------------------------------------------------------
\section{Objetivos}
% --------------------------------------------------------

Seus objetivos...


\subsection{Subseção}

\lipsum[3-56]

\subsubsection{Subsubseção}

\lipsum[3-56]

\subsubsubsection{Subsubsubseção}

\lipsum[3-56]

%Essa dissertação tem como objetivo principal propor uma aplicação para aquisição e autenticação dos dados de consumo gerados por SMs usando uma nuvem computacional.

%Com objetivos secundários:

%\begin{itemize}
%	\item Verificar a pertinência do uso da arquitetura SGX de forma a satisfazer os requisitos de segurança da informação;

%	\item Propor um abordagem para garantir a medição até o faturamento;
    
 %   \item Anonimizar os dados de consumo por uma terceira parte confiável;

%\end{itemize}
% ---

%\section{Relevância}
% ---
%Os desafios de segurança para aplicações em nuvem de coleta de dados, ainda são problemas abertos, carente de soluções definitivas.

%O trabalho colaborá diminuindo os obstáculos em aplicações de computação em nuvem empregadas em adquirir e processar dados de medidores inteligentes. Contribuindo com adoção célere de REI seguras.
% ---

\section{Organização do Trabalho}
% ---
Esta seção do trabalho apresenta a estrutura no qual o trabalho está organizado.

O capitulo 2 apresenta as bases teorias da literatura mundial que concerne esta pesquisa. Tem bases teóricas referentes a computação em nuvem, \textit{Smart Grids}, \textit{Smart Meters}, microsserviços e a tecnologia Intel SGX. 

O capitulo 3 apresenta trabalhos relacionados a esta pesquisa. Os trabalhos serão referentes a segurança em ambientes em computação em nuvem e em mecanismos de segurança utilizando a plataforma Intel SGX.

%O capítulo 7 apresenta o cronograma para término dessa dissertação de mestrado.

% ---

%\include{capitulos/cap02-Conceitos_Basicos}
%\include{capitulos/cap03-trabalhos_relacionados}
%\include{capitulos/cap04-materias_metodos}
%\include{capitulos/cap05-validacao}
%\include{capitulos/cap06-conclusao}
%\include{capitulos/cap07-cronograma}


% PARTE
%\part{Proposta}
%\include{capitulos/proposta}

% PARTE
%\part{Parte Final}
%\include{capitulos/resultados}
%\include{capitulos/conclusao}

% ----------------------------------------------------------
% ELEMENTOS PÓS-TEXTUAIS (Referências, Glossário, Apêndices)
% ----------------------------------------------------------
\postextual

% Referências bibliográficas
\bibliography{bibliografia}

% Glossário (Consulte o manual)
%\glossary

% Apêndices
%% ----------------------------------------------------------
% Apêndices
% ----------------------------------------------------------

% ---
% Inicia os apêndices
% ---
\begin{apendicesenv}

% Imprime uma página indicando o início dos apêndices
\partapendices

% ----------------------------------------------------------
\chapter{Códigos do RSA dentro do ENCLAVE}
\label{ap cod rsa enclave}
% ----------------------------------------------------------

/App/App.cpp

\#include <stdio.h>\\
\#include <iostream>\\
\#include "Enclave\_u.h"\\
\#include "sgx\_urts.h"\\
\#include "sgx\_utils/sgx\_utils.h"\\
\#include <time.h>\\
\#include <unistd.h>\\ 
\#include "mbedtls/pk.h"\\
\#include "mbedtls/entropy.h"\\
\#include "mbedtls/error.h"\\
\#include "glue.h"\\
\#include <string.h>\\
\#include <stdlib.h>\\
\\

/* Global EID shared by multiple threads */\\
sgx\_enclave\_id\_t global\_eid = 0;\\

// OCall implementations\\
void ocall\_print(const char* str) {\\
    printf("\%s$\backslash$n", str);\\
}\\






%--------------------------------------------------------
\end{apendicesenv}
% ---

% Anexos
%\include{postextual/anexos}

% Índice remissivo (Consultar manual)
%\phantompart
%\printindex

\end{document}
% * <marcosaraujo.mvma@gmail.com> 2017-04-10T12:18:58.497Z:
%
% ^.
%  Configurações de aparência do PDF final
% NÃO ALTERAR!!!

% alterando o aspecto da cor azul
\definecolor{blue}{RGB}{41,5,195}

% informações do PDF
\makeatletter


%-----------------------------------------------------------------------------
% CONFIGURACAO DO SUMARIO
%-----------------------------------------------------------------------------
% Modifica o espaçamento no sumário
% Nao ha espacos, exceto para as entradas de capitulos
\setlength{\cftbeforeparagraphskip}{0pt}
\setlength{\cftbeforesubsectionskip}{0pt}
\setlength{\cftbeforesectionskip}{0pt}
\setlength{\cftbeforesubsubsectionskip}{0pt}
\setlength{\cftbeforechapterskip}{\onelineskip}

% Alteração da indentação dos itens do sumário

\cftsetindents{section}{20pt}{30pt}
\cftsetindents{subsection}{30pt}{40pt}
\cftsetindents{subsubsection}{40pt}{50pt}
\cftsetindents{paragraph}{50pt}{60pt}



% seção primária (Chapter) Caixa alta (alterado dentro de \makeatletter), negrito
\renewcommand{\cftchapterfont}{\bfseries}

% seção secundária (Section) Caixa alta (alterado dentro de \makeatletter)
\renewcommand{\cftsectionfont}{\normalfont}

% seção terciária (Subsection) caixa baixa, negrito
\renewcommand{\cftsubsectionfont}{\normalfont\bfseries}

% seção quaternária (Subsubsection) Caixa baixa, itálico
\renewcommand{\cftsubsubsectionfont}{\itshape}

% seção quinária (subsubsubsection) Caixa baixa
\renewcommand{\cftparagraphfont}{\normalfont}

%-----------------------------------------------------------------------------


\hypersetup{
     	%pagebackref=true,
		pdftitle={\@title}, 
		pdfauthor={\@author},
    		pdfsubject={\imprimirpreambulo},
	    pdfcreator={LaTeX with abnTeX2},
		pdfkeywords={abnt}{latex}{abntex}{abntex2}{trabalho acadêmico}, 
		colorlinks=true,       		% false: boxed links; true: colored links
    		linkcolor=black,          	% color of internal links
    		citecolor=black,        		% color of links to bibliography
    		filecolor=black,      		% color of file links
		urlcolor=blue,
		bookmarksdepth=4
} 
\makeatother
% --- 
